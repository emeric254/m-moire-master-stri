\selectlanguage{french}
\begin{abstract}

Ce mémoire relate mes deux années d'apprentissage au sein du groupe Orange au poste d'architecte logiciel dans le programme API.
Le travail de ce programme consiste à promouvoir la conception selon des méthodes orientées services dans le cycle de vie du système d'information de l'entreprise, principalement en suivant les concepts d'API REST.
Les problématiques de gestion des systèmes d'information sont le nerf de la guerre des multinationales et ce sont le cœur du travail des directions des systèmes d'informations.
Des styles d'architectures logiciels tels que le REST ou les micro-services permettent de concevoir et d'améliorer la qualité des systèmes d'information dans le but de réduire les risques et les coûts.
Cependant certaines contraintes freinent encore leur adoption et leur mise en place dans cette entreprise : les infrastructures actuelles ne permettent pas encore l'obtention de tous les gains de ces architectures lorsque ce ne sont pas directement les différentes personnes des projets qui rejettent ces nouveautés.
Ces styles d'architectures ne répondent pas à tous les besoins mais ils sont une base solide et prête à être améliorés dans la construction d'un système d'information complexe et pérenne.
De nouveaux concepts et de nouvelles architectures se basent directement sur celles-ci et étendent leurs fonctions pour répondre à de nouveaux besoins toujours plus forts de communications inter-services internes et externes aux entreprises.

\end{abstract}

%
\selectlanguage{english}
\begin{abstract}

This report presents my two year apprenticeship in Orange as a software architect for the API program.
The API program promotes new service oriented architectures to information systems, mostly REST API.
Main concerns in firms is information systems complexity and its ability to be upgrade, these are real business risks and costs.
Some architecture designs like REST or micro-services solve these issues by providing universal way to conceive simple and consistent application interfaces.
They are a good start to continue extending information systems while limiting its complexity sum.
Nowadays there is a trouble in Orange, there is problems with current infrastructure to support these new architectures and their advantages are not totally perceived.
New concepts and architectures based on actual ones are already available to improves further functions and gains from this base.
Main goal is yet to made information systems able to handle always more requests between services inside and outside of companies.

\end{abstract}
