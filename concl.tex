\section*{Conclusion}
\addcontentsline{toc}{section}{Conclusion}

Ces deux années d'apprentissage m'ont permis d'affiner ma compréhension globale de l'entreprise : sa hiérarchie, son fonctionnement, ses processus et le cloisonnement entre les différents domaines métiers (branches grand public, professionnel et B2B).
Pendant ces périodes en entreprises j'ai aussi appréhendé une multitude de métiers tant techniques que relationnels et commerciaux.

Aujourd'hui les systèmes d'informations sont devenus trop complexes pour une gestion sans risques opérationnels dans toutes les évolutions.
Certains styles d'architectures permettent de répondre à ces problématiques de gestion de la complexité des SI, comme par exemple les API REST.

Les services que j'ai rendus auprès des différents projets m'ont permis de comprendre l'importance de l'outillage simple mais spécifique.
Les prestations internes sont absolument nécessaires dans l'organisation et la réponse aux besoins des différents services, ce sont également des éléments indispensables pour le développement et la montée en compétences des salariés de l'entreprise.

Avec ce Master, je retiens qu'une très bonne maîtrise des outils informatiques et ses évolutions est indispensable dans tous les métiers.
De même, une bonne gestion du système d'information est l'élément clef de la réussite d'une multinationale.
L'alternance est une opportunité d'entrée dans le monde professionnel qui apporte aux étudiants la vision de salarié, complémentaire à notre formation.

%
%~ \subsection{Une sous section}

%~ On peut mettre des mots en \emph{italique},
%~ en \textsc{petites Majuscules} ou
%~ en \texttt{largeur fixe (machine à écrire)}.

%~ Voici un deuxième paragraphe avec une formule mathématique simple : $e = mc^2$.

%~ Un troisième avec des \og guillemet français \fg{}.

%~ \foreignlanguage{english}{Do you speak French? Does anybody here speak french?}

%~ \begin{itemize}
%~ \item Liste classique ;
%~ \item un élément ;
%~ \item et un autre élément.
%~ \end{itemize}
%~ \vspace{\parskip} % espace entre paragraphes

%~ \begin{enumerate}
%~ \item Une liste numéroté
%~ \item deux
%~ \item trois
%~ \end{enumerate}
%~ \vspace{\parskip}

%~ \begin{description}
%~ \item[Description] C'est bien pour des définitions.
%~ \item[Deux] Ou pour faire un liste spéciale.
%~ \end{description}
%~ \vspace{\parskip}
